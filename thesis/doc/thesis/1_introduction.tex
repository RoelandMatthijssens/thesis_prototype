%!TEX root = thesis.tex 

\chapter{Introduction} \label{cha:Introduction}
\paragraph{} 
We are currently living in an information age where an estimated \num{150000} new websites are created on a daily basis. Since the HTML 2.0 standard was introduced in 1995, content creators are able to add anchor tags to their websites. These tags allow web pages to be interlinked, forming a vast web of information. To navigate this Web, we use browsers such as Google Chrome\footnote{\url{http://www.google.com/chrome/}}, Mozilla Firefox\footnote{\url{http://www.mozilla.org/firefox/}}, Safari\footnote{\url{http://www.apple.com/safari/}} or Oracle's Opera\footnote{\url{http://www.opera.com}}.

\paragraph{} 
The HTML standard has gone through several iterations since HTML 2.0. We are currently at HTML5 and a lot of features have been changed and added. However, the way web pages are linked together has remained unchanged. The main mechanism web developers use to create links between web pages is still the same anchor tag that was introduced almost 20 years ago. This anchor tag is not without limitations though and the only type of links that can be established using anchor tags are unidirectional single source and single destination links between two resources.

\paragraph{} 
 One possible definition for the Web that we could define is the following: \dquote{\emph{A system of interlinked hypertext documents accessed via the Internet}}. The core of this definition is that these pages need to be interlinked, therefore it is rather strange that the technology that is used to create these links has not been updated or improved over the last decades.

\paragraph{} 
HTML and HTTP were invented in 1990 by Sir Timothy John Berners-Lee~\cite{fielding1998hypertext}. The term hypertext was already coined in 1965 by Ted Nelson~\cite{myers1998brief}, predating the HTML standard by a couple of decades. Nelson defines hypertext as \dquote{\emph{Non-sequential writing}} meaning that the text can branch in different directions. At a certain point in a block of text, a pointer may be introduced that directs the readers to a different section, or even a different document. Ted Nelson's idea of hypertext was heavily influenced by Vannevar Bush and his idea of the Memex~\cite{Bush1945As}. With his hypertext, Nelson aims to mimic the associative links between content as introduced by Bush. Ted Nelson's hypertext idea may sound reminiscent of the current HTML's anchor tags, where one source can embed a link to another source. But in fact Ted Nelson stated that his vision of hypertext actually tried to prevent the evolution that is currently active in HTML standards~\cite{nelson1987computer}. HTML only supports unidirectional links, and as a result, quotes cannot be followed back to their original source. There is also no notion of version or rights management and existing links keep breaking when the content of websites is updated. Nelson stated, that due to these limitations the current HTML standard is a major oversimplification of his original idea of hypertext and hypermedia\footnote{\url{http://hyperland.com/TedCompOneLiners}}.

\paragraph{} 
These obvious problems with linking in HTML were the cause of several initiatives attempting to \dquote{fix the web} and get closer to the initial ideas of hypertext and hypermedia. Most of these initiatives originate from other research labs that aim to enhance the Web. The main difference between existing tools and our tool is the setting in which it will be used. Existing tools were designed and developed as a private tool to be used by a single person for their own personal interests. At best, these tools could be used by small groups of users sharing the same interests and needs. The solution we will present in this thesis is designed and developed as a tool for communities or even large sets of different communities with different interests and needs. However, in designing this tool, we encountered an important problem. This problem was less obvious in existing tools because they were intended to be used by a single person which, in most cases, greatly limits the amount of data the tool would have to operate on. When we increased the amount of users that will use the tool simultaneously, we encountered some interesting issues.

\paragraph{} 
Imagine users annotating a web page with comments and adding a links between two web pages they have previously visited. Users will highlight sentences or words on these pages to mark the location of the annotation and the sources of the links. Because they are the only users using the system, they can choose to not create too many hyperlinks on a specific paragraph to keep the content readable. They will also adopt a specific scheme of annotating that will suit their work style and keep the web pages usable. In other words, the user can easily control the amount of metadata that is put on a specific page. If a page gets too cluttered for the users' taste, they can remove or rewrite some of the annotations until the web page is to their liking once more. In a multi-user setting, this issue becomes a lot more prominent. Imagine a popular news website that gets thousands of daily hits. A good percentage of these users will read an article about something they have read before. These users will share their information with the rest of the community by adding an annotation to the article or by adding a link to another website that has more information on the topic. Since there are thousands of users reading this specific article, it is possible that too many users try to share their information, thus quickly cluttering the text with metadata. If we naively add all this extra data to the web page the page will become unreadable and the tool will hinder the users instead of helping them find additional information. For most of the existing tools, the usability of the tool is therefore inversely proportional to the amount of users that will share their data with each other.

\paragraph{} 
The main problem when a multitude of users are sharing the same database is actually a visualisation problem. The usability issues that other tools have, originate from their naive way of visualising the metadata on the web page. The question that arises is the following: \dquote{\emph{How can we make use of the metadata that is generated by a large community of users, without reducing the usability of the web page?}} To answer this question we need to explore several different aspects of the problem.
\begin{itemize}
	\item \textit{Filtering} How can we distil the important and relevant data out of the vast amounts of generated metadata in order to help users find what they are looking for more quickly?
	\item \textit{Visualising} How will we visualise the selected metadata in such a way that the users have a clear overview of the available information without making the web page unreadable?
	\item \textit{Interaction} With techniques to create more complex hyperlinks with multiple destinations and multiple sources we need to provide more powerful interactions for the users to browse this more complex Web.
\end{itemize}

\paragraph{} 
In this thesis we explore previous attempts to increase the Web's linking functionality and investigate areas where improvements can be made. In Chapter \ref{cha:Limitations of Linking Techniques}, we discuss the current techniques that are used to link resources. We proceed by showing why these techniques are suboptimal and why there is a need for improved linking functionality. In Chapter \ref{cha:Towards a Better Web}, we take a look at alternative approaches to the default HTML linking. These alternative tools give us additional insights on how we could improve upon the default linking functionality.

\paragraph{} 
Based on our conclusions from Chapter \ref{cha:Towards a Better Web} and combined with our personal insights, we define the key concepts that our tool needs to provide in Chapter~\ref{cha:An Alternative Approach}. By generalising our problem to a data visualisation problem instead of limiting us to the canvas of a web page, we can apply recent work in the active field of data visualisation. These insights help us to mitigate some of the problems alternative approaches introduced. In Chapter \ref{cha:The Tool}, we propose a different solution based on our findings in Chapter \ref{cha:An Alternative Approach}. This solution is then implemented and discussed in Chapter \ref{cha:Implementation}. In order to demonstrate the benefits that our solution claims to provide, we follow up the implementation with use cases that will show how the tool operates in a real world setting. This is done in Chapter \ref{cha:Use Cases} where we walk through the entire process of authoring and navigating through annotations and hyperlinks, discussing our improvements along the way.

\paragraph{} 
As this is a Master's thesis, time was limited and our work is restricted by the available time. However, in Chapter \ref{cha:Conclusions and Future Work} we provide our conclusions as well as a discussionof some possible future work that could be carried out to further extend the tool.

\paragraph{} 
By extending the Web's current linking technology and by allowing communities to use a single database to share their metadata, we aim to provide a better web experience for both content creators and consumers. We are confident that our proposed tool improves many of the issues we have identified and greatly improves the accessibility of metadata for large groups of users.
