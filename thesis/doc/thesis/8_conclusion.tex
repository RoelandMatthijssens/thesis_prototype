%!TEX root = thesis.tex 

\chapter{Conclusions and Future Work} \label{cha:Conclusions and Future Work}
\paragraph{}
In this final chapter, we would like to recapitulate a number of contributions that have been made. As we were limited by the time constraint of this thesis, some  envisioned features were not yet implemented. Therefore we will outline and discuss some future directions for our linking tool.

\section{Contributions} \label{sec:Contributions}
\paragraph{}
The literature study that has been carried out as part of this thesis shows that there is room for improvements in the area of hyperlinking on the Web. The market of web browsers is heavily dominated by classic browsers such as Internet Explorer, Google Chrome and Firefox and these browsers have heavily influenced the Web's linking standard.
\paragraph{}
While there are alternative approaches available, these tools often specialise on specific features instead of the linking process as a whole. Many of these tools also neglect the upcoming trend of sharing user created data. Nevertheless, by studying these tools, we learned a few valuable lessons concerning the ways of enhancing the Web's linking experience and how they relate to the limitations of the current web standard.
\paragraph{}
By combining the different aspects that we deem beneficial for our tool with our own ideas for an ideal tool, we came up with a model and architecture for an advanced linking tool addressing many of the issues the current state-of-the-art hyperlinking tools. Our tool steps away from the classic restrictions of the two dimensional web page and is based on state-of-the-art research from fields such as hypermedia, zoomable user interfaces and data visualisation in order to redesign the way we use hyperlinks on the Web.
\paragraph{}
Our investigation of the state-of-the-art in hypermedia, data visualisation and zoomable user interfaces helped us to avoid some of the pitfalls when designing out tool. What follows is a description of the steps we took, and contributions we made during this thesis
\begin{itemize}
	\item A literature study has been done in order to identify issues with current linking tools. We compared the benefits of state-of-the-art linking tools to the standard HTML approach in order to use our findings in our work.
	\item Based on our findings from the literature study and a closer investigation of existing tools, we identified technologies that might be beneficial in the creation of an advanced hyperlinking tool and help us to overcome some of the limitations these tools suffer. Based on this analysis we listed the benefits and drawbacks of a large amount of options we have for our tool and based on this list we created a design for our ideal linking tool. After careful deliberation regarding the use of concepts and technologies, we provided a solution that mitigates or eliminates many of the issues and shortcomings with existing tools.
	\item Based on our analysis of the limitations of state-of-the-art web linking tools we created a set of requirements for our tool and we defined a model and architecture for our proposed solution.
	\item Finally we implemented a functional prototype of our tool which uses many of the innovative features and forms the basis of our advanced hyperlinking tool.
\end{itemize}
Because of our requirement to build the tool using JavaScript only, we opted to build the tool from scratch. This allowed us to leave conventional approaches behind. Doing so allowed us to keep in mind the core requirements when building the foundations of the tool which inevitably leads to a better designed tool that provides to most functionality and extensibility.
\section{Future Work} \label{sec:Future Work}
\paragraph{}
Because of the limited time available for this thesis,  some extensions and features did not make it to the final implementation, thus the tool has not yet reached its full potential. Therefore we made sure that we finished a solid foundation on which future features can be implemented. We did not rush the implementation in favour of excessive features and different visualisations, instead we opted to ensure that we got the basics right and modular from the beginning so that extensions are easily implemented. As a result we did not have time left to implement many different visualisations which would emphasise the power of our tool. In this section, we discuss some extensions and future work that will help bring our tool a step closer to its full potential.
\section{Robustness} \label{sec:Robustness}
\paragraph{}
One of the big limitations of our tool is the suboptimal performance on highly dynamic web pages. When the content of the web page changes while the page's identifier stays the same, it is possible that the selectors of previously saved hyperlinks will have a mismatch on the content of the page. This will of course break existing links, should the content of the web page change, but it also makes the tool not viable on websites with highly dynamic pages such as most of the Google web services. Detecting whether the content of the web page has changed since the time the hyperlink was created can be done by hashing the web page when the hyperlink is created and then matching this hash with the hash of the current content on the page. When the hashes do not match we have detected that the content has changed.
\paragraph{}
One way to minimise the amount of breaking hyperlinks is by detecting whether the content on which the hyperlink was created is still available. The content of the web page might have moved around a bit in such a way that the XPointer of the related selector is no longer valid, but the content is still there. If we can detect this, many hyperlinks that would otherwise break, can still be valid. Imagine a blogging website where each new post is listed at the top, moving all previous post down on the page. Hyperlinks created on any of these blog posts would break every time a new post was created with the current method, but if we check the content of the web page, these hyperlinks will be valid for a substantially longer period of time.
\section{Interactivity} \label{sec:Interactivity}
\paragraph{}
We are aware that we did not yet devote a lot of attention to editing and removing existing hyperlinks or selectors which would lead to more technical problems when the database grows. We have already implemented a means of detecting when hyperlinks are no longer deemed interesting by the community and this metric can be used to automatically remove hyperlinks from the database. But when the users make a mistake and want to edit or remove their hyperlinks they are currently unable to do so.
\paragraph{}
We are currently heavily reliant on the users' capability to create a selection on a web page, but on many new devices this is not necessarily straightforward or easy. Many smartphones or tablets do not allow users to easily create selections in their web browsers, which limits the tools usefulness on these devices. With better interactive possibilities to create selectors, the tool could be used on more devices and by more users.
\paragraph{}
When new filters and visualisations get introduced this will bring the need to manipulate these visualisations as well, introducing the need for more advanced interactivity.
\section{Visualisation} \label{sec:Visualisation}
\paragraph{}
The biggest trump of our tool is the versatility with which the metadata can be visualised. Many diverse questions can be easily answered with different visualisations and for a further extension of the tool many new interesting and useful ways of data visualisation can be added. Because of our limited time we only opted to implement some of the more obvious visualisations but many unused dimensions of the metadata can still be presented to the users in such a way that the information is easy accessible.
\section{General} \label{sec:General}
\paragraph{}
As we can see, most of the work carried out in this thesis covers the foundation of an extendible tool for advanced hyperlinking. In future work, additional research could be done for more diverse visualisations and interactions. Additional research regarding the filtering techniques can also increase the effectiveness of the tool and by extension the usefulness for the end users. Finally, including additional media types and the support for transclusion, will elevate the tool to a true collaborative cross-media hyperlinking tool that scales well with the amount of users.
\paragraph{}
Even though we were able to identify the many shortcomings of current state-of-the art hyperlinking tools, propose innovative features for non-restricted hyperlink visualisation and provide a prototype implementation of a next generation hyperlinking tool, our tool has to be seen as the foundation of an extendible collaborative hyperlinking solution.
