%!TEX root = thesis.tex 

\chapter*{Abstract}
\thispagestyle{empty}
\paragraph{}
With over \num{150000} new websites being produced every day, the World Wide Web is playing an increasingly more important role in day to day life. From personal use to big Internet companies, the Web is omnipresent in a modern society. The Web was created in the late 1980s and has been publicly available since 1991. However, after all this time, very little has changed to the core ideas. The idea of the Web was seeded by Vannevar Bush's vision of the Memex but has evolved hastily to match the needs of the consumers. This haphazard evolution of the Web gave priority to new flashy features neglecting the foundation of the Web: Linking different resources together. Modern websites are still mostly using simple anchor tags, a technique designed in 1994, to link their web pages together.
\paragraph{}
In this thesis, we take a critical stance toward conventional linking techniques and the popular tools that have become a standard for browsing the Web. From an evolutionary point of view, it is clear why the current standard includes outdated techniques. Even though anchor tags are the single most used technique to link resources together, it does not imply that this is the most optimal or complete approach.
\paragraph{}
Through literature study and critical thinking, we identify a number of important flaws and shortcomings regarding the core concepts of standard linking techniques on the Web. A closer look at alternative approaches showed that more advanced and complete approaches are already available. By studying these alternatives and taking a step back from the dogma of standard linking techniques, we were able to apply state-of-the-art research from the fields of information management and visualisation, effectively increasing the Web's potential and ameliorating the majority of its current limitations. Our proposed solution provides a new angle on the way resources can be linked on the Web and bases itself on recent research in the areas of hypermedia and data visualisation.
\paragraph{}
Separating the linking metadata from the structure of the web page minimises the misapplication of markup languages which are currently used to link different resources together. This separation will pave the way for more complex linking structures which will in turn provide increased functionality on the Web. The increased functionality allows users to share their metadata with each other which is currently a dominant evolution on the Web. User communities will fill large databases with associative links between different web resources, sharing their knowledge and insights. In time the amount of metadata that is generated will be of such large proportions that adding all this additional data to a web page will make it unusable and unreadable. We combine the research from the field of data management and the findings from our literature study to propose a scalable solution to these newly introduces issues. Our solution also separates the visualisation of the metadata from the web page allowing us to employ innovative techniques from the fields of data visualisation and interactive user interfaces. Furthermore, the use of the latest HTML5 standards for the visualisation and JavaScript for the implementation ensures minimum coupling with maximum compatibility.
\paragraph{}
While it is perfectly possible to create classic hyperlinks with our solution, it offers features that go beyond state-of-the-art linking techniques. Our solution provides the means to step away from the unidirectional single source and single target hyperlinks that traditional linking solutions provide. The choice of technology allows a degree of interactivity that is impossible with existing tools providing interactive filtering, navigation and searching as well as many other features that are unheard of in current state-of-the-art linking solutions.



%Through our innovative MindXpres solu-
%tion, we are going to enhance the presentation experience for the presenter
%and audience alike in the near future!
\newpage
